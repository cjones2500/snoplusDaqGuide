%%%%%%%%%%%%%%%%%%%%%%%%%%%%%%%%%%%%%%%%%
%----------------------------------------------------------------------------------------
%	PACKAGES AND DOCUMENT CONFIGURATIONS
%----------------------------------------------------------------------------------------

\documentclass[12pt,preprint]{article}

\usepackage{mhchem} % Package for chemical equation typesetting
\usepackage{siunitx} % Provides the \SI{}{} command for typesetting SI units
\usepackage{setspace} 
\usepackage{graphicx} % Required for the inclusion of images
\usepackage{float}
%\usepackage{natbib}

\setlength\parindent{0pt} % Removes all indentation from paragraphs

\renewcommand{\labelenumi}{\alph{enumi}.} % Make numbering in the enumerate environment by letter rather than number (e.g. section 6)

\headsep = 0pt
\topmargin = 0pt
\headheight = 0pt
\textwidth =450pt
\marginparsep = 10pt
\oddsidemargin = 30pt
%\usepackage{times} % Uncomment to use the Times New Roman font

%----------------------------------------------------------------------------------------
%	DOCUMENT INFORMATION
%----------------------------------------------------------------------------------------

\title{SNO+ Guide to the DAQ} % Title

\author{Christopher \textsc{Jones}, Andy Mastbaum, Richie Bonventre} % Author name

\date{\today} % Date for the report

\begin{document}

\maketitle % Insert the title, author and date

% If you wish to include an abstract, uncomment the lines below
\begin{abstract}
\noindent
A guide to understanding the entire structure of the SNO+ DAQ, designed for EXPERT Detector Operators 
\end{abstract}

\section{Introduction}\normalsize
Introduction to the DAQ (I think it would be nice to explain from a simple event in the detector how light detected on a PMT becomes a signal sent to dataflow)

\section{Orca}


\section{MTC/D}


\section{MTC/A+}
I propose to firstly calibrate the charge by increasing how much we sample the charge slopes and feeding this back into the electronics calibration. 

\section{XL3}

\section{Front End Electronics}



\section{Builder}

\section{ELLIE System}

\section{Slow Control}

\section{Jargon Dictionary}
\begin{itemize}
\item Jargon jargon jargon 
\end{itemize}

\section{Links to relevant documents}


%\section{Future work} 

%\begin{thebibliography}{99}

%\bibitem[1]{Beacom}
%Dasgupta, B. \& Beacom, J., F., Physical Rev. D, 83, 113006, 2011

%\bibitem[2]{O'Keeffe}
%O'Keeffe, H.,M., O'Sullivan, E., \& Chen, M., Nucl. Instrum. Meth. A, 1:5, 2011

%\bibitem[3]{Schumaker}
%Schmaker, M.A., SNO+ Collaboration, Nuc. Phys. B Proceed. Suppl. 1:1, 2010

%\bibitem[4]{dasgupta}
%Dasgupta, B., Dighe, A. \& Mirizzi, A., PRL, 101, 171801, 2008

%\bibitem[5]{fogli}
%Fogli, G., Lisi, E., Marrone, A., \& Mirizzi A., Journ. Cosm. Astropart. Phys. 12010, 2007

%\end{thebibliography}
%----------------------------------------------------------------------------------------
%	BIBLIOGRAPHY
%----------------------------------------------------------------------------------------

%\bibliographystyle{unsrt}

%\bibliography{sample}

%----------------------------------------------------------------------------------------


\end{document}